\documentclass[a4paper,twoside,nofonts]{tufte-handout}

\usepackage[french,english]{babel}
\usepackage[utf8]{inputenc}
\usepackage[T1]{fontenc}

\usepackage[default]{lato}
\usepackage{lato-latex}
% \usepackage{PTSerif,PTSans}
% \renewcommand{\familydefault}{\sfdefault}
\usepackage[scaled=1.1]{inconsolata}
\usepackage{amsmath}
\usepackage{ccicons}

\usepackage{multicol}
\setlength\columnsep{1.5em}
\usepackage{graphicx}
\usepackage{hyperref}
\definecolor{internallinkcolor}{named}{violet}
\definecolor{externallinkcolor}{named}{teal}
\hypersetup{
  breaklinks,
  colorlinks,
  urlcolor=externallinkcolor,
  linkcolor=internallinkcolor,
  filecolor=externallinkcolor,
  citecolor=internallinkcolor,
}
\usepackage{microtype}

\usepackage{csquotes}
\usepackage{soul,soulutf8}
\usepackage{multirow}
\newcommand\multirowSeparatorRight[2][\quad]{%
  \multirow{#2}*{\color{gray}$#1\blacktriangleright #1$}}

\usepackage{listings}
\lstset{
  basicstyle={\ttfamily},
  columns=fullflexible,
  language=[LaTeX]TeX,
  escapeinside={_}{_},
  escapebegin={\normalfont},
  escapeend={},
  gobble=2
}
\lstMakeShortInline| % typeset sequences |like this| as code
\lstnewenvironment{latexcode}{%
  \lstset{basicstyle=\small\ttfamily}%
}{}

\newcommand\parambegin{\normalfont{\rmfamily\textlangle}\itshape}
\newcommand\paramend{\normalfont{\rmfamily\textrangle}}
\newcommand\parameter[1]{{\parambegin#1\paramend}}
\newcommand\code[1]{\texttt{#1}}
\let\file\code

\def\resultstyle{%
  \def\rmdefault{cmr}%
  \def\ttdefault{cmtt}%
  \def\sfdefault{cmss}%
  \let\familydefault\rmdefault
  \normalfont}
\newcommand\result[1]{{\resultstyle #1}}
\newenvironment{longresult}{\resultstyle\small}{\par}

\usepackage{todo-setup}
\usepackage{lipsum}

\raggedbottom
\setcounter{tocdepth}{1}

\makeatletter
\def\wikipedi@prefix{http://en.wikipedia.org/wiki/}
\def\ctan@prefix{http://ctan.tug.org/pkg/}
\newcommand\wikipedia[2][\@empty]{%
  \href{\wikipedi@prefix #2}{\ifx #1\@empty #2\else #1\fi}}
\newcommand\package[2][\@empty]{%
  \href{\ctan@prefix #2}{\file{\ifx #1\@empty #2.sty\else #1\fi}}}
\makeatother


%%%%%%%%%%%%%%%%%%%%%%%%%%%%%%%%%%%%%%%%%%%%%%%%%%%%%%%%%%%%%%%%
\title{\LaTeX'ing with style}
\author{Damien Pollet}
\date{}

\begin{document} %%%%%%%%%%%%%%%%%%%%%%%%%%%%%%%%%%%%%%%%%%%%%%%
\maketitle

\begin{abstract}
  A few pieces of advice for \LaTeX{} users, to help them prepare documents with better typography, streamline their writing workflow, and write maintainable \LaTeX{} code.
\end{abstract}
\begin{todoenv}
  \noindent Similar things:
  \url{http://www.charlietanksley.net/philtex/essential-packages/}
\end{todoenv}
\vfill

\begin{fullwidth}
  \begin{multicols}{2}
    \noindent
    \LaTeX{} is well-known for the quality of the documents it produces, but to achieve really nice results, authors still have to be familiar with good typography practices.
    Even so, there is the problem of finding just \emph{how} to implement some details properly, without getting swamped in crazy syntax, or spending too many hours looking for the providential package and neurons debugging obscure macros.
    
    I have tried to gather general advice for knowledgeable \LaTeX{} users who want to achieve better typographic results or more maintenable source code, but who do not have the time to go through the litterature and the piles of documentation on the path to that goal.
    So this is neither technical documentation, nor a typography course ---I left that in the aforementioned piles--- but rather a set of directions of where the nice things are.
    Be aware, then, that what follows is \emph{my own} travel map, and that I traced it following my own influences, preferences, and intuitions; it's not absolute, and certainly not exhaustive.
    Rules are meant to be broken; follow your own path!
  \end{multicols}
  \bigskip
  
  \begin{multicols}{3}
    \noindent
    First, \emph{\nameref{sec:typography}} is about the readers: typography, graphics, layout and how to convey your message efficiently.
    
    \columnbreak\noindent
    \emph{\nameref{sec:tools}} is about you, the author, and how to setup your environment to make the writing process as painless as possible.
    
    \columnbreak\noindent
    Finally, \emph{\nameref{sec:code}} is about your co-authors as well as future-you: how to make the \LaTeX{} code itself readable and easy to work with.
  \end{multicols}
\end{fullwidth}



\clearpage
\section{Pleasant typography} % (fold)
\label{sec:typography}

%\cite{bringhurst,chicago}

\subsection{Within the paragraph} % (fold)
\label{sec:paragraph}

\paragraph{Do not abuse of fonts.}
With fonts, less is more.
\marginnote{%
  A \wikipedia{font} means a single size, style and weight of a particular typeface.
  Conversely, a \wikipedia{typeface} or font family is a set of fonts with stylistic unity.}
\LaTeX{} makes it tempting to abuse of the many complete typefaces available, but emphasizing everything has the opposite effect: nothing really stands out anymore.

Normal paragraph text only needs a roman font and matching italic for emphasis.
It's often said that foreign locutions like \enquote{i.e.} or \enquote{et al.} should be emphasized but I think it's useless because they are so common in scientific English.
Technical publications often use a sans-serif or monospaced font as well, to distinguish formal identifiers from homonymous domain terms.

For emphasis, use the |\emph{}| command.
Italic is the preferred way to emphasize text, because it's designed to make text stand out in the reading flow, but not in the page; conversely, bold text stands out visually by disturbing the typographic gray, so it should only be used when that is the intent ---dictionary entries, for instance.
As for underlining, it is just a workaround to replace proper italics where they are not possible, like typewriters or handwriting; you can safely forget that it even exists.

\paragraph{Meaningful spacing.}
Whitespace might not use any ink, but it does convey meaning.
\Todo{small fig: vertical/horizontal grids}
Depending on how it's used, whitespace can either separate or attach elements; all depends on the quantity of space between two elements, relative to other nearby spaces.

Take the time to use non-breaking spaces where necessary, and to adjust its size accordingly.
For instance, the non-breaking space in |Section~\ref{}| is absolutely necessary to prevent the number from appearing alone at the beginning of the next line.
For the abbreviated versions, I like to switch to a half-width space, because they read more like an identifier than two words, for instance:
\begin{center}\small
  \begin{tabular}{ r c l }
    |Sect.\,\ref{}| & \multirowSeparatorRight{2}
    & \result{Sect.\,3.14}
    \\
    |p.\,\pageref{}| &
    & \result{p.\,42}
  \end{tabular}
\end{center}

\paragraph{Language conventions.}
I always load \package{babel}, even in purely english documents; at the minimum, this ensures that words are correctly hyphenated.
In non-English or multilingual documents, you should load all necessary languages, the last one being the main one, for instance |\usepackage[french,english]{babel}|.
Then, switch languages using the |otherlanguage| environment, or even inline:
\begin{latexcode}
  \foreignlanguage{french}{Bonjour} means \emph{hello}.
\end{latexcode}

Besides hyphenation, the other major feature of \package{babel} is that it enforces language-specific typographic conventions, like spacing around punctuation signs.
If you have lots of quotations, rather than using |``explicit quotes''|, load \package{csquotes} and use |\enquote{automatic quotes}|.

\paragraph{Small caps acronyms.}
Acronyms in full caps are fine, but can be disturbing if they appear too frequently.
One way to alleviate that is setting them like proper nouns, but that is only possible if the acronym is pronounced like a word, rather than spelled; periods are superfluous in either case.
Otherwise, small caps are the solution.
\marginnote{%
  Proper small caps are \emph{not} scaled-down upper-case letters.
  They have to be designed to match the line weight and general proportions of normal letters.}
If your text font has proper small caps, they will make acronyms less obtrusive; \package{soul} provides the |\caps{}| command, which will even slightly letterspace its argument, for aesthetic purposes.
Note that in the code, you should write the acronym in lower case; |\caps| will conserve upper-case letters, which can come handy if you need to begin a sentence by an acronym.
\begin{multicols}{2}%
  \selectlanguage{french}%
  \begin{longresult}\noindent
    La notation UML est née de la fusion de OMT, Booch et OOSE.
  \end{longresult}
  \columnbreak
  \begin{longresult}\noindent
    La notation \caps{uml} est née de la fusion de \caps{omt}, Booch et \caps{oose}.
  \end{longresult}
\end{multicols}

\paragraph{Coordinate typefaces.}
Many document classes mix text typefaces that were not designed together.
You should of course pick typefaces with compatible aesthetic temperaments, but also ensure that they coordinate well visually.
For instance, many well-known publication templates use Times together with Helvetica, but while both have the same point size, they do not \emph{look} like it.
\marginnote{\wikipedia{X-height} matching is quite sensitive; discrepancies of a few percent easily stand out to the attentive eye.}
Fortunately, \package{helvet} and a few other typeface packages can adjust the scale of the font; pick the value to match the x-heights:
\begin{latexcode}
  \usepackage[scaled=0.82]{helvet}
\end{latexcode}


\paragraph{Keep body text black.}
To be readable, small text needs good contrast without texture noise.
Unfortunately, the halftoning patterns of most black and white laser printers are not fine enough to produce comfortably readable body text; even very dark colors produce noticeably jagged edges.
For documents that will be printed, I would thus advise keeping all body text pure black.

For documents intended to be read mainly on screen, you could adopt the |colorlinks| option of \package{hyperref}, which is less intrusive than the colored boxes, but I still think the default colors call for some tweaking\sidenote[][-\baselineskip]{%
  Compared to black, we want colors subtle enough to act like italics, indicating the presence of a link, without calling attention to it like bold would.}.
Below is how I set it up in this document; have a look at the documentation for \package{xcolor} for the list of predefined colors.
\begin{latexcode}
  \definecolor{internallinkcolor}{named}{violet}
  \definecolor{externallinkcolor}{named}{teal}
  \hypersetup{
    breaklinks, colorlinks,
    urlcolor=externallinkcolor,
    filecolor=externallinkcolor,
    linkcolor=internallinkcolor,
    citecolor=internallinkcolor,
  }
\end{latexcode}

\begin{todoenv}
    fonts, emphasis
    source code, program elements
    ligatures and spaces
    links, colors
    language, babel
\end{todoenv}
% (end)

\subsection{Graphics, tables, \& other floats} % (fold)
\label{sec:graphics}

\cite{tufte}

\begin{todoenv}
    graphicx
    spacing
    rules, booktabs
    listings setup
\end{todoenv}
% (end)

\subsection{Document layout} % (fold)
\label{sec:layout}

\cite{bringhurst,designingbooks}

\begin{todoenv}
    line length
    spacing
    proportions, your own document style
\end{todoenv}
% (end)
% section typography (end)



\clearpage
\section{Efficient tools} % (fold)
\label{sec:tools}

\subsection{Compiling \& previewing} % (fold)
\label{sec:compiling}

\begin{todoenv}
    latexmk + extensions
    synctex
    spell-checking?
\end{todoenv}
% (end)

\subsection{Learning \& debugging} % (fold)
\label{sec:learning}

\begin{todoenv}
  texdoc
  kpsewhich
  kpsepath
\end{todoenv}
% (end)

\subsection{Bibliography} % (fold)
\label{sec:bibliography}

\begin{todoenv}
    key format
    keywords
    url, doi…
\end{todoenv}
% (end)

\subsection{Version control} % (fold)
\label{sec:vcs}

\begin{todoenv}
    bbl
    one sentence per line
    git primer ?
\end{todoenv}
% (end)

\subsection{Maintenance} % (fold)
\label{sec:maintenance}

\begin{todoenv}
    todo notes
    document templates
\end{todoenv}
% (end)
% section tools (end)



\clearpage
\section{Nice code} % (fold)
\label{sec:code}

\subsection{Semantics \& form} % (fold)
\label{sec:semantics}

\begin{todoenv}
    encoding
    references
    define environments for examples, code etc
    product / team names, recurrent things
    managing the final adjustments
\end{todoenv}
% (end)

\subsection{Organizing \& presenting \LaTeX{} code} % (fold)
\label{sec:format}

\begin{todoenv}
    presenting main text code
    one sentence per line
    macros: indentation, commenting whitespace
    preamble stuff
    big documents
\end{todoenv}
% (end)
% section code (end)



\clearpage
\bibliographystyle{plainnat}
\nocite{*}
\bibliography{local}

\vfill
\begin{fullwidth}\centering
  \ccbysa\quad
  This work is licensed under a \href{http://creativecommons.org/licenses/by-sa/3.0/}{Creative Commons Attribution-ShareAlike 3.0 License}.
\end{fullwidth}
\clearpage\todos

\end{document}
% Local Variables:
% coding: utf-8-unix
% LaTeX-indent-level: 2
% TeX-brace-indent-level: 2
% auto-complete-mode: t
% End:
